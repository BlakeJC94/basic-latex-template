% ================================================================
% LOAD PACKAGES

% Default AMS packages
\usepackage{
  amsmath,  % Extended equation environments
  amssymb,  % Extended maths symbols (also loads amsfonts)
  amsthm    % Theorem-like structures
} 

% Graphics packages
\usepackage{
  caption,    % Write captions 
  graphicx,   % Include graphics (with extras)
  pgfplots,   % For charts
  tikz,       % Draw tikz pictures
  subcaption  % Write subcaptions
}
\pgfplotsset{compat=1.13}  % set version of pgfplots

% Formatting packages
\usepackage{
  bookmark,    % Better bookmark management
  datetime,    % Output date and time 
  easyReview,  % Reviewer commands (\replace{.}{.}, \remove{.}, \add{.})
  enumitem,    % Better control over lists (used for to-do lists)
  fancyhdr,    % Extended header and footer options 
  hyperref,    % Generate hyperlinks for references
  multicol,    % Multicolumn environment
  natbib,      % Better inline citations (\citep, \citet, \citeauthor)
  pifont,      % Extra symbols (used for check boxes in to-do env)
  setspace     % Simple linespacing (\doublespacing)
}


% ================================================================
% DECLARE THEOREM ENVIRONMENTS
\theoremstyle{plain}
  \newtheorem{theorem}{Theorem}[section]
  \newtheorem{lemma}[theorem]{Lemma}
  \newtheorem{proposition}[theorem]{Proposition}
  \newtheorem{corollary}[theorem]{Corollary}
\theoremstyle{definition}
  \newtheorem*{definition}{Definition}
  \newtheorem{remark}[theorem]{Remark}
  \newtheorem{exercise}[theorem]{Exercise}
  \newtheorem{problem}{Problem}
  \newtheorem{solution}{Solution}
  \newtheorem{example}[theorem]{Example}
    

% ================================================================
% DECLARE MATHS MACROS 

% Common symbols
\def\vep{\varepsilon}           % \vep - Script eps
\def\del{\partial}              % \del Partial derivative
\def\P{\mathbb{P}}              % \P - Probability measure
\def\E{\mathbb{E}}              % \E - Expectation operator
\def\Var{\text{Var}}            % \Var - Variance operator
\def\Cov{\text{Cov}}            % \Cov - Covariance
\def\F{\mathcal{F}}             % \F - Sigma-field
\def\Tr{\text{Tr}}              % \Tr - Matrix trace
\def\Det{\text{Det}}            % \Det - Matrix determinant
\def\grad{\boldsymbol{\nabla}}  % \grad - Gradient operator
\def\lap{\nabla^2}              % \lap - Laplacian
\def\I{\mathbb{I}}              % \I - Indicator 

% Common structures
\newcommand{\br}[1]{\left \lbrace #1 \right \rbrace}     % \br{.} - braces
\newcommand{\abs}[1]{\left \vert #1 \right \vert}        % \abs{.} - abs val
\newcommand{\floor}[1]{\left \lfloor #1 \right \rfloor}  % \floor{.} - floor
\newcommand{\ceil}[1]{\left \lceil #1 \right \rceil}     % \ceil{.} - ceiling 
\newcommand{\norm}[1]{\left \|#1 \right \|}              % \norm{.} - norm
\newcommand{\vc}[1]{\boldsymbol{#1}}                     % \vc{.} - vector (bb)

\newcommand{\ncr}[2]{  % \ncr{}{} - Combination
  \left(               % e.g. \ncr{25}{3}
    \begin{matrix} 
      #1 \\ #2 
    \end{matrix} 
  \right)
}

\newcommand{\ddx}[3][]{  % \ddx[]{}{} - Derivative operator 
  \frac{d^{#1} {#2}}     % e.g. \ddx{f}{x}, \ddx[2]{f}{x}
    {d{#3}^{#1}}  
} 

\newcommand{\ppx}[3][]{      % \ppx[]{}{} - Partial derivative operator 
  \frac{\partial^{#1} {#2}}  % \ppx{f}{x}, \ppx[2]{f}{x}
    {\partial{#3}^{#1}}      % TODO: update command for mixed partials
}


% ================================================================
% DOCUMENT FORMATTING

% Design header and footer
\fancyhf{} 
\rfoot{Page \thepage}

% Clear default bib heading (required for multi column bib)
\renewcommand\bibsection{}

% Define \outline{} for drafting structure
\newcommand{\outline}[1]{
  \ifx\draft\undefined 
  \else 
    \if\draft1 
      \noindent [#1] 
    \fi 
  \fi
}


% Define date format for final
\newdateformat{monthyeardate}{
  \monthname[\THEMONTH], \THEYEAR
}

% Define date format for draft
\newdateformat{draftdate}{
  \currenttime 
  \ifx\TIMEZONE\undefined
  \else
    \space \TIMEZONE 
  \fi
  \space - \space 
  \ordinal{DAY} of \monthname[\THEMONTH], 
  \THEYEAR
}


% Define to-do list environment
\newlist{todolist}{itemize}{2}
\setlist[todolist]{label=$\square$}
% Define ticks and crosses symbols
\newcommand{\cmark}{\ding{51}}%
\newcommand{\xmark}{\ding{55}}%
% Define items ticks and crosses for to-do env
\newcommand{\done}{%
  \rlap{$\square$}{\raisebox{2pt}{\large\hspace{1pt}\cmark}}%
  \hspace{-5.5pt}
}
\newcommand{\wontfix}{%
  \rlap{$\square$}{\large\hspace{1pt}\xmark}%
}
% example code:
% \begin{todolist}
%   \item[\done] Frame the problem
%   \item Write solution
%   \item[\wontfix] profit
% \end{todolist}
% source: https://tex.stackexchange.com/a/247688

